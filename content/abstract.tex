% => Wenn die Arbeit auf Deutsch verfasst wurde, verlangt das Studienreferat KEINEN englischen Abstract

% % englischer Abstract
%\null\vfil
%\begin{otherlanguage}{english}
%\begin{center}\textsf{\textbf{\abstractname}}\end{center}
%
%\noindent Lorem ipsum dolor sit amet, consetetur sadipscing elitr, sed diam nonumy eirmod tempor invidunt ut labore et dolore magna aliquyam erat, sed diam voluptua. At vero eos et accusam et justo duo dolores et ea rebum. Stet clita kasd gubergren, no sea takimata sanctus est Lorem ipsum dolor sit amet. Lorem ipsum dolor sit amet, consetetur sadipscing elitr, sed diam nonumy eirmod tempor invidunt ut labore et dolore magna aliquyam erat, sed diam voluptua. At vero eos et accusam et justo duo dolores et ea rebum. Stet clita kasd gubergren, no sea takimata sanctus est Lorem ipsum dolor sit amet.
%
%\end{otherlanguage}
%\vfil\null


% => Wenn die Arbeit auf Englisch verfasst wurde, verlangt das Studienreferat einen englischen UND deutschen Abstract (der dt. Abstract kann dann ggf. auch ans Ende der Arbeit)

% deutsche Zusammenfassung
\null\vfil
\begin{otherlanguage}{ngerman}
\begin{center}\textsf{\textbf{\abstractname}}\end{center}

\noindent Das Projekt Natur 4.0 Sensing Biodiversity zielt darauf ab, zum Naturschutz ein Umweltmonitoringsystem aufzubauen. Dazu wird ein automatisches Modell zur visuellen Erkennung von Tierarten benötigt. Damit ein solches Modell im Rahmen dieser Arbeit entwickelt werden konnte, wurden zwei Deep-Learning-Modelle vom Typ Convolutional Neural Network verwendet: Der MegaDetector für die Tierlokalisierung und ein EfficientNet-Modell für die Tierklassifizierung. Der MegaDetector wurde zuvor fertig trainiert und konnte daher sofort zum Einsatz kommen, aber das beste EfficientNet-Modell zur Tierklassifizierung musste im Rahmen dieser Arbeit bestimmt werden. Dies geschah, indem die potenziellen EfficientNet-Modelle zur Tierklassifizierung zuerst auf demselben Trainingsdatensatz trainiert und danach auf demselben Testdatensatz getestet wurden. Dazu wurden Trainings- bzw. Testbilder erworben und mithilfe des MegaDetectors verarbeitet, bevor sie den potenziellen EfficientNet-Modellen übergeben wurden. Die Testergebnisse zeigten, dass das EfficientNet-B4 mit einer Top-1 Accuracy von \textbf{92,9\%} das beste Modell zur Tierklassifizierung im Rahmen dieser Arbeit war. Da aber ungenügende Daten für die Leistungsevaluierung des MegaDetectors verfügbar waren, war es unmöglich, die Leistungsfähigkeit des Gesamtmodells zur Tierarterkennung zu bewerten.

\end{otherlanguage}
\vfil\null



